\documentclass[a4paper]{scrartcl}
\usepackage[utf8]{inputenc}% muss zum Editor passen -> http://texwelt.de/wissen/fragen/2656/
\usepackage[T1]{fontenc}
\usepackage[ngerman]{babel}
\usepackage{multicol}
\usepackage{wallpaper}
% \usepackage{hyperref} 

\usepackage{geometry}
\geometry{a4paper,left=25mm,right=25mm, top=25mm, bottom=3cm}

\renewcommand{\familydefault}{\sfdefault}

\title{mMail: Sicherheit im Mailverkehr durch Verschlüsselung}
\author{\normalsize{WagnerTech UG, Turfstr. 18a, 81929 München, www.wagnertech.de}}
\date{\vspace{-6ex}}

\ULCornerWallPaper{1}{wagner_tech_briefbogen_blau_fs1.pdf}

\begin{document}
\pagenumbering{gobble} % no page numbers

\maketitle

\begin{abstract}
Nach dem Bekanntwerden der geheimdienstlichen Tätigkeiten im Internet ist es Allgemeingut geworden, dass sensible
Firmendaten im ungeschützten Internet nichts zu suchen haben. Wie soll aber eine Firma im täglichen Betrieb für
die nötige Sicherheit sorgen, ohne den freien Informationsaustausch in unnötiger Weise zu behindern? Dieser Artikel
zeigt, wie mit entsprechenden Einstellungen im \emph{mail transfer agent} (MTA) \emph{postfix} für eine den 
Anforderungen eines Unternehmens graduelle Sicherheit realisiert werden kann.
\end{abstract}

\begin{multicols}{2}
Eine Sicherung des Mailverkehrs durch Verschlüsselung kann auf verschiedenen Ebenen erfolgen. Als sicherste Lösung gilt
dabei die Ende-zu-Ende-Verschlüsselung über ein \emph{public key}-Verfahren. Bei diesem Verfahren werden die \emph{e-mails}
mit dem öffentlichen Schlüssel des Empfängers\footnote{Die grammatikalisch männliche Form umfasse sämtliche 
Geschlechtsausprägungen. Daher wird diese Form Formulierungen vorgezogen, die sich auf zwei Geschlechter beschränken. 
Bei allen anderen Ausdrücken wird in derselben Weise verfahren.} 
verschlüsselt. Nur der Empfänger der Mail kann die 
so verschlüsselte Mail mit seinem privaten Schlüssel wieder entschlüsseln. Dieses Verfahren hat folgende Nachteile:
	\begin{itemize}
	\item Alle Teilnehmer müssen die öffentlichen Schlüssel ihrer Kommunikationspartner auf ihrem Endgerät zur Verfügung
		haben.
	\item Ein Teilnehmer muss auf jedem seiner Endgeäte (PC, Laptop, Smartphone) seinen privaten Schlüssel zur Verfügung
		haben.
	\item Die Vervielfältigung des privaten Schlüssels ist eine potentielle Sicherheitslücke, wenn bei der Übertragung
		ungeeignete Methoden (z.B. Verschicken mit Mail, Clouddienste) verwendet werden.
	\end{itemize}
Eine andere Form der Verschlüsselung ist die Transportverschlüsselung. Hier werden die Daten zwischen MTAs über das 
ssl-Protokoll verschlüsselt ausgetauscht. Dieses Verfahren hat folgende Nachteile:
	\begin{itemize}
	\item In den Mailfächern der Server liegen die Daten in unverschlüsselter Form vor.
	\item Ein Austausch über Transportverschlüsselung darf daher nur zu MTAs hin erfolgen, auf denen die (unverschlüsselten)
		Mails sicher liegen. Also nicht zu einem der großen Provider, auf die interessierte Stellen Zugriff haben.
	\end{itemize}
mMail ist ein \emph{postfix}-Zusatz, der folgenden Ansatz verfolgt:
	\begin{itemize}
	\item Der Mailverkehr innerhalb der Firma kann unverschlüsselt erfolgen.
	\item Für den Mailverkehr zu Partnerfirmen genügt eine Transportverschlüsselung.
	\item Mailverkehr zu einer Einzeladresse (z.B. Privatadresse eines Mitarbeiters) darf nur über eine Ende-zu-Ende
		Verschlüsselung erfolgen.
	\item Die Ver- und Entschlüsselung für das \emph{public key}-Verfahren erfolgt dabei zentral auf dem Firmen-Mail-Server,
		damit der einzelne Mitarbeiter nicht mit der Schlüsselverwaltung zu tun hat.
	\end{itemize}
mMail ist \emph{open source} und kann über unsere Firma kostenlos bezogen werden. WagnerTech unterstützt Sie gerne bei
Einrichtung und Betrieb der Software.

\end{multicols}

\end{document}
